\section{Transformări liniare}
\newtheorem{th_linear_transforms}{Definiția}
\begin{th_linear_transforms}
Fie V și W două spații vectoriale. O funcție $L : V \rightarrow W$ se numește
transformare liniară dacă verifică relațiile :
\begin{easylist}
\ListProperties(Margin=1cm,Hang1=true)
@ $L(v_{1} + v_{2}) = L(v_{1}) + L(v_{2})$, pentru oricare doi vectori 
    $v_{1}, \quad v_{2} \in V$.
@ $L(\alpha v) = \alpha L(v)$, pentru orice vector $v \in V$ și orice scalar 
$\alpha$.    
\end{easylist}
\end{th_linear_transforms}
Transformările liniare sunt convenient de utilizat pentru că au o reprezentare
care le face ușor de implementat și calculat într-un program. Când cele două
spații vectoriale V și W sunt \textit{finit dimensionale}, o transformare 
liniară poate fi exprimată ca o multiplicare de matrici. 
Fie $\{v_{1}, \dots , v_{m}\}$ și $\{w_{1}, \dots , w_{n}\}$ bazele ortonormate ale
spațiilor vectoriale V și W și între care există o operație de produs interior. 
Atunci, matricea corespunzătoare transformării T este :
\begin{equation}
A = (a_{ij}) = \langle w_{i}, T(v_{j}) \rangle
\end{equation}
Dacă vectorul $w \in W$ este rezultatul aplicării transformării liniare $L$ 
asupra vectorului $v \in V$ iar $A$ este matricea acestei transformări, mai
putem scrie :
\begin{equation}
w = Av
\end{equation}
Când cele două spații vectoriale $V, W$ au aceeași dimensionalitate este posibil
ca transformarea $L$ să fie inversabilă, adică să existe o transformare $L^{-1}$
astfel încât $LL^{-1} = I$.