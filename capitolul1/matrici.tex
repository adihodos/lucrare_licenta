\section{Matrici}
\label{ch1:sec_matrix}

Într-un engine grafic, calculele pot fi realizate într-o multitudine de spații
de coordonate Carteziene diferite. Trecerea de la un spațiu de coordonate la
altul presupune utilizarea matricilor de transformare. Așadar, matricile sunt un
concept fundamental pentru orice aplicație grafică, iar scopul acestui capitol
este de a explora proprietățile lor.

\subsection{Proprietățile matricilor}
\label{ch1:sec_matrix:sub_matrix_properties}

\newtheorem{th_matrixdef}{Definiția}
\begin{th_matrixdef}
Fie $a_{ij}, i = 1, 2, \dots m, j = 1, 2, \dots n, m \times n$ numere. Se
numește matrice $m \times n$ tabloul dreptunghiular
$
A =
\begin{Vmatrix}
a_{11} & a_{12} & \dots & a_{1n} \\
a_{21} & a_{22} & \dots & a_{2n} \\
\dots \\
a_{n1} & a_{n2} & \dots & a_{mn}
\end{Vmatrix} = \Vert a_{ij} \Vert
$ cu m linii și n coloane. Numerele $a_{ij}$ se numesc elementele matricii.
\end{th_matrixdef}

Două matrice $m \times n \mathbf{A} = \Vert a_{ij} \Vert, \quad 
\mathbf{B} = \Vert b_{ij} \Vert$ sunt egale dacă $a_{ij} = b_{ij}, \quad 
\forall i,j \in {1, 2, \dots n}$. Operația de egalitate a matricilor este
definită numai dacă cele două matrici au același număr de linii și coloane.

\newtheorem{th_matrixadd}[th_matrixdef]{Definiția}
\begin{th_matrixadd}
Suma a două matrice $m \times n$, $\mathbf{A} = \Vert a_{ij} \Vert, \quad
\mathbf{B} = \Vert b_{ij} \Vert$ este matricea $A + B = \Vert a_{ij} + b_{ij}
\Vert$.
\end{th_matrixadd}

Proprietățile adunării matricilor.
\begin{enumerate}
  \item Adunarea este comutativă : $\mathbf{A + B = B + A}$, deoarece
  $\mathit{a_{ij} + b_{ij} = b_{ij} + a_{ij}}$.
  \item Adunarea matricilor este asociativă : $\mathbf{A + (B + C) = (A + B) +
  C}$, deoarece $\mathit{a_{ij} + (b_{ij} + c_{ij}) = (a_{ij} + b_{ij}) +
  c_{ij}}$.
  \item Adunarea matricilor admite element neutru, matricea \textbf{O}
  (matricea nulă), cu m linii și n coloane, adică $\mathbf{A + O = O + A = A}$.
  \item Orice matrice are un opus, adica $\forall A = \Vert a_{ij} \Vert \quad
   \exists \quad (-A) = \Vert (-a_{ij}) \Vert \quad \text{a.î} \quad 
   A + (-A) = (-A) + A = O$.
\end{enumerate}
Aceste proprietăți arată că mulțimea matricilor de $m \times n$ linii și
coloane cu elemente din $\mathbf{R, C}$ formează un grup abelian față de
operația de adunare.
