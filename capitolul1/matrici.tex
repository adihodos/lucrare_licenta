\section{Matrici}
\label{ch1:sec_matrix}

Într-un engine grafic, calculele pot fi realizate într-o multitudine de spații
de coordonate Carteziene diferite. Trecerea de la un spațiu de coordonate la
altul presupune utilizarea matricilor de transformare. Așadar, matricile sunt un
concept fundamental pentru orice aplicație grafică, iar scopul acestui capitol
este de a explora proprietățile lor.

\subsection{Proprietățile matricilor}
\label{ch1:sec_matrix:sub_matrix_properties}

\newtheorem{th_matrixdef}{Definiția}
\begin{th_matrixdef}
Fie $a_{ij}, i = 1, 2, \dots m, j = 1, 2, \dots n, m \times n$ numere. Se
numește matrice $m \times n$ tabloul dreptunghiular
$
A =
\begin{Vmatrix}
a_{11} & a_{12} & \dots & a_{1n} \\
a_{21} & a_{22} & \dots & a_{2n} \\
\dots \\
a_{n1} & a_{n2} & \dots & a_{mn}
\end{Vmatrix} = \Vert a_{ij} \Vert
$ cu m linii și n coloane. Numerele $a_{ij}$ se numesc elementele matricii.
\end{th_matrixdef}

Două matrice $m \times n \mathbf{A} = \Vert a_{ij} \Vert, \quad 
\mathbf{B} = \Vert b_{ij} \Vert$ sunt egale dacă $a_{ij} = b_{ij}, \quad 
\forall i,j \in {1, 2, \dots n}$. Operația de egalitate a matricilor este
definită numai dacă cele două matrici au același număr de linii și coloane.

\newtheorem{th_matrixadd}[th_matrixdef]{Definiția}
\begin{th_matrixadd}
Suma a două matrice $m \times n$, $\mathbf{A} = \Vert a_{ij} \Vert, \quad
\mathbf{B} = \Vert b_{ij} \Vert$ este matricea $A + B = \Vert a_{ij} + b_{ij}
\Vert$.
\end{th_matrixadd}

Proprietățile adunării matricilor.
\begin{enumerate}
  \item Adunarea este comutativă : $\mathbf{A + B = B + A}$, deoarece
  $\mathit{a_{ij} + b_{ij} = b_{ij} + a_{ij}}$.
  \item Adunarea matricilor este asociativă : $\mathbf{A + (B + C) = (A + B) +
  C}$, deoarece $\mathit{a_{ij} + (b_{ij} + c_{ij}) = (a_{ij} + b_{ij}) +
  c_{ij}}$.
  \item Adunarea matricilor admite element neutru, matricea \textbf{O}
  (matricea nulă), cu m linii și n coloane, adică $\mathbf{A + O = O + A = A}$.
  \item Orice matrice are un opus, adică $\forall A = \Vert a_{ij} \Vert \quad
   \exists \quad (-A) = \Vert (-a_{ij}) \Vert \quad \text{a.î} \quad 
   A + (-A) = (-A) + A = O$.
\end{enumerate}
Aceste proprietăți arată că mulțimea matricilor de $m \times n$ linii și
coloane cu elemente din $\mathbf{R, C}$ formează un grup abelian față de
operația de adunare.

Înmulțirea a două matrici dreptunghiulare $A \times B$, unde
\begin{center}
$A = \Vert a_{ij} \Vert, \quad i = 1, 2, \dots n, \quad j = 1, 2, \dots m$, \\
$B = \Vert b_{hk} \Vert, \quad h = 1, 2, \dots p, \quad k = 1, 2, \dots q$
\end{center}
nu este definită decât dacă $p = n$, adică numărul coloanelor matricii A este
egal cu numărul liniilor matricii B. Dacă această condiție este îndeplinită, produsul
$A \times B$ este o matrice $A \times B = \Vert c_{ij} \Vert$, unde $c_{ij} =
\sum_{h = 1}^{n}a_{ih}b_{hk}$, deci $A \times B$ este o matrice de $m \times q$.

Înmulțirea unei matrici cu un scalar $\lambda$ rezultă intr-o matrice $\lambda A
= \Vert \lambda a_{ij} \Vert$.

O matrice pentru care numărul de linii este egal cu numărul de coloane se
numește matrice pătratică.

Oricare ar fi trei matrici pătratice de ordinul n, $A, B, C \in \mathrm{R}_{n}$
sunt adevărate următoarele afirmații :
\begin{enumerate}
  \item Adunarea este comutativă, adică : $A + B = B + A$
  \item Adunarea este asociativă, adică : $A + (B + C) = (A + B) + C$
  \item Matricea \textbf{O} de ordinul n, având toate elementele egale cu zero,
  este element neutru pentru adunare, adică $A + O = O + A = A$
  \item Pentru orice matrice \textbf{A} din $\mathrm{R}_n$ există opusul
  $\mathbf{-A} = \Vert -a_{ij} \Vert$, astfel încât $A + (-A) = (-A) + A = O$.
  \item Înmulțirea matricilor este asociativă, adica $A \times (B \times C) =
  (A \times B) \times C$
  \item Elementul neutru în $\mathfrak{R}^n$ este matricea unitate de ordinul
  \textit{n}, notată cu $\mathbf{I}_{n}$
  \[
  \mathbf{I}_{n} = \Vert \delta_{ij} \Vert,
  \begin{cases}
  0, \quad i \neq j \\
  1, \quad i = j
  \end{cases} \quad \text{sau} \quad
  \mathbf{I}_{n} = 
  \begin{vmatrix}
  1 & 0 & 0 & \dots & 0 \\
  0 & 1 & 0 & \dots & 0 \\
  0 & 0 & 1 & \dots & 0 \\
  \hdotsfor{5} \\
  0 & 0 & 0 & \dots & 1
  \end{vmatrix}
  \] și pentru care avem $A \times I_{n} = I_{n} \times A = A$.
  \item Înmulțirea matricilor este distributivă față de operația de adunare,
  adică $A \times (B + C) = A \times B + A \times C$.
\end{enumerate}

\todo{Inserat transpusa matricii aici, proprietătile determinantilor fac
referire la tanspusa}

\newtheorem{th_transpose}[th_matrixdef]{Definiția}
\begin{th_transpose}
Fie $A = \Vert a_{ij} \Vert$ o matrice de \textit{m} linii și
\textit{n} coloane. Se numește transpusa matricii A și se noteaza cu $A^T$,
matricea de \textit{n} linii și \textit{m} coloane și care verifică relația
$a_{ij} = a^T_{ji}, \forall i = 1 \to m, j = 1 \to n$.
\end{th_transpose}
Pentru transpusa unei matrici sunt valabile următoarele proprietăți :
\begin{enumerate}
  \item $(A^T)^T = A$ (transpusa transpusei este matricea inițială).
  \item $(A + B)^T = A^T + B^T$.
  \item $(A \times B) ^ T = B ^ T \times A ^ T$.
\end{enumerate}

O matrice se numește simetrică dacă elementele sale verifică relația $a_{ij} =
a_{ji}$. Transpusa unei matrici simetrice este egală cu matricea inițială.
Exemplu : matricea
\begin{equation*}
A = 
\begin{bmatrix}
1 & 9 & 4 \\
9 & 5 & 3 \\
4 & 3 & 7  
\end{bmatrix}
\end{equation*} este o matrice simetrică, deoarece elementele ei verifică
cerința ca $a_{ij} = a_{ji}$.
\indent
 
O matrice ale cărei elemente verifică relația $a_{ij} = -(a_{ji})$ se numește
matrice antisimetrică. O matrice antisimetrică are toate elementele de pe 
diagonala principală egale cu zero (din condiția $a_{ij} = -(a_{ji})$).
Transpusa unei matrici antisimetrice este egală cu opusul matricii inițiale,
adică $A^T = -(A)$.
Exemplu : matricea
\begin{equation*}
A = 
\begin{bmatrix}
0 & 9 & -4 \\
-9 & 0 & -5 \\
4 & 5 & 0
\end{bmatrix}
\end{equation*} este o matrice antisimetrică, iar transpusa ei este :
\begin{equation*}
A^T =
\begin{bmatrix}
0 & -9 & 4 \\
9 & 0 & 5 \\
-4 & -5 & 0
\end{bmatrix}.
\end{equation*}

\newtheorem{th_det}[th_matrixdef]{Definiția}
\begin{th_det}
Fie A o matrice pătratică din $\mathfrak{R}_{n}$. Se numește determinantul
matricii A și se notează cu 
\[
detA = 
\begin{vmatrix}
a_{11} & a_{12} & \dots & a_{1n} \\
a_{21} & a_{22} & \dots & a_{2n} \\
\hdotsfor{4} \\
a_{n1} & a_{n2} & \dots & a_{nn}
\end{vmatrix}, \quad \text{scalarul dat de formula}  
\sum_{\substack{\sigma \in S_{n}}} \epsilon (\sigma) a_{1\sigma(1)}
a_{2\sigma(2)} \cdots a_{n\sigma(n)}
\], unde $S_{n}$ este mulțimea tuturor permutărilor de gradul $\mathit{n}$ și
$\epsilon(\sigma)$ este signatura permutării $\sigma$.
\end{th_det}
Pentru o excelentă introducere în permutări, lucrările \cite{NitaNS96} sau
\cite{Rosculet87} sunt recomandate.

Pentru a ușura calculul determinanților, ne putem folosi de o serie de
proprietăți ale acestora :
\begin{enumerate}
  \item Determinantul unei matrice coincide cu determinantul transpusei matricii
  respective, adică $det(A) = det(A^{T}), \forall A \in \mathfrak{R}_{n}$.
  \item Dacă într-o matrice toate elementele unei linii sau coloane sunt egale
  cu zero, atunci determinantul matricii este nul.
  \item Dacă într-o matrice schimbăm între ele două linii sau două coloane,
  obținem o matrice al cărei determinant este egal cu opusul determinantului
  matricii inițiale.
  \item Dacă o matrice are două linii sau două coloane identice, atunci
  determinantul său este nul.
  \item Dacă toate elementele unei linii (sau coloane) ale unei matrici sunt
  înmulțite cu un număr $\alpha$, obținem o matrice al cărei determinant este
  egal cu $\alpha$ înmulțit cu determinantul matricei inițiale.
  \item Dacă într-o matrice elementele a două linii sau două coloane sunt
  proporționale, atunci determinantul matricii este nul.
  \item Dacă o linie sau o coloană este o combinație liniară de celelalte linii
  sau coloane, atunci determinantul matricii este nul.
  \item Dacă la o linie sau coloana a unei matrice adunăm elementele altei linii
  sau coloane, înmulțite cu același număr, obținem o matrice al cărei
  determinant este egal cu determinantul matricei inițiale.
  \item $det(A \times B) = det(A) \times \det(B), \forall A, B \in
  \mathfrak{R}_{n}$.
\end{enumerate}
