\section{Vectori}
\label{ch1:sec_vectors}
\indent

Vectorii au o importanță fundamentală pentru orice aplicație 3D. Ei sunt
utilizați pentru a reprezenta puncte in spațiu, precum locațiile obiectelor sau
verticele unei meșe de triunghiuri. De asemenea sunt utilizați pentru a
reprezenta direcții în spațiu, precum orientarea camerei sau normalii la
suprafață ai unei meșe de triunghiuri.

\subsection{Proprietățile vectorilor}
\label{ch1:sec_vectors:sub_properties}
\indent 

Deși sunt posibile definiții mai abstracte, ne vom concentra asupra vectorilor
definiți prin \textit{n}-tuple de numere reale, unde \textit{n} este de obicei
2, 3 sau 4. Un vector \textit{n}-dimensional \textbf{V} poate fi scris ca
\begin{equation}
\label{eq:vec:1}
\mathbf{V} = \langle V_1, V_2, \dots V_n \rangle,
\end{equation}  unde numerele $\mathit{V_i}$ se numesc componentele vectorului
\textbf{V}.
Vectorul din ecuația \eqref{eq:vec:1} poate fi reprezentat și printr-o matrice
cu o singură coloană și \textit{n} linii :
\begin{equation}
\label{eq:vec:2}
\mathbf{V} = 
\begin{bmatrix}
V_1 \\
V_2 \\
\vdots \\
V_n
\end{bmatrix}.
\end{equation} 
În mod normal vectorii vor fi exprimați în acest fel, uneori însă apare nevoia
de a exprima un vector ca și o matrice cu o singură linie și \textit{n} coloane.
Vom scrie un vector linie ca și transpusul vectorului coloană corespunzător :
\begin{equation}
\label{eq:vec:3}
\mathbf{V^T} = 
\begin{bmatrix}
V_1, \quad V_2, \quad \dots \quad V_n
\end{bmatrix}.
\end{equation}

\indent
 Un vector poate fi multiplicat cu un scalar pentru a produce un
nou vector ale cărui componente au aceeași proporție relativă. Produsul dintre
scalarul \textit{a} și vectorul \textbf{V} este definit ca :
\begin{equation}
\label{eq:vec:4}
\mathit{a}\mathbf{V} = \mathbf{V}\mathit{a} = \langle \mathit{a}V_1, \quad
\mathit{a}V_2, \quad \dots \quad \mathit{a}V_n \rangle.
\end{equation}

\indent
Vectorii se adună și se scad pe componente. Dîndu-se doi vectori \textbf{P} și
\textbf{Q}, definim suma lor \textbf{P + Q} ca :
\begin{equation}
\label{eq:vec:5}
\mathbf{P + Q} = \langle P_1 + Q_1, \quad P_2 + Q_2 \quad \dots \quad P_n + Q_n
\rangle.
\end{equation}
Diferența a doi vectori, scrisă ca și \textbf{P - Q} este doar o simplificare a
notației pentru suma \textbf{P + (-Q)}.\\
Vom examina acum cîteva proprietăți fundamentale ale aritmeticii cu vectori.

%\theoremstyle{definition} 
\newtheorem{vector_arith}{Teorema}

\begin{vector_arith}
Oricare ar fi doi scalari \textit{a} și \textit{b} cu \textit{a}, \textit{b}
$\in \mathbf{R}$ și oricare trei vectori \textbf{P, Q, R} cu $\mathbf{P},
\mathbf{Q}, \mathbf{R} \in \mathbf{R^n}$, următoarele proprietăți sunt
adevărate.
\begin{enumerate}
  \item $\mathbf{P + Q = Q + P}$ - adunarea este este comutativă.
  \item $\mathbf{(P + Q) + R = P + (Q + R)}$ - adunarea este asociativă.
  \item $\mathit{(ab)}\mathbf{P} = \mathit{a(b}\mathbf{P}\mathit{)}$ -
  asociativitatea înmulțirii cu scalari.
  \item $\mathit{a(\mathbf{P + Q})} = \mathit{a}\mathbf{Q} +
  \mathit{b}\mathbf{Q}$ - înmulțirea cu scalari este distributivă față de
  adunarea vectorilor.
  \item $\mathit{(a + b)}\mathbf{P} = \mathit{a}\mathbf{P} +
  \mathit{b}\mathbf{P}$ - înmulțirea vectorilor este distributivă față de
  adunarea scalarilor.
\end{enumerate}
\end{vector_arith}

\indent
\textit{Magnitudinea} unui vector \textit{n}-dimensional \textbf{B} este un
scalar notat cu $\Vert \mathbf{V} \Vert$ și este dat de formula :
\begin{equation}
\label{eq:vec:6}
\Vert \mathbf{V} \Vert = \sqrt{\sum_{i = 1}^{n} V_i^2}.
\end{equation}
Magnitudinea unui vector se mai numește și \textit{norma} sau \textit{lungimea}
unui vector. Un vector cu o magnitudine de unu se spune ca are \textit{lungime
unitate}, sau că este un \textit{vector unitate}. Atunci cînd \textbf{V}
reprezintă un vector din $\mathbf{R^3}$, equația \eqref{eq:vec:6} poate fi
scrisă ca :
\begin{equation}
\label{eq:vec:7}
\Vert \mathbf{V} \Vert = \sqrt{V_x^2 + V_y^2 + V_z^2}.
\end{equation}

\indent
Un vector care are cel puțin o componentă non-zero poate fi redimensionat la
lungimea unitate prin multiplicare cu $\frac{1}{\Vert \mathbf{V} \Vert}$.
Această operație se mai numește și normalizare.

\indent
Magnitudinea unui vector are următoarele proprietăți.
\newtheorem{mag_vectors}[vector_arith]{Teorema}
\begin{mag_vectors}
Oricare ar fi scalarul $\mathit{a} \in \mathbf{R}$ și oricare ar fi doi vectori
$\mathbf{P, Q} \in \mathbf{R^n}$, următoarele proprietăți sunt adevărate.
\begin{enumerate}
  \item $\Vert \mathbf{P} \Vert \geq 0$
  \item $\Vert \mathbf{P} \Vert = 0 \iff \mathbf{P} = \langle 0, \quad 0, \quad
  \dots \quad 0 \rangle$
  \item $\Vert \mathit{a}\mathbf{P} \Vert = \vert \mathit{a} \vert \Vert
  \mathbf{P} \Vert$
  \item $\Vert \mathbf{P + Q} \Vert \leq \Vert \mathbf{P} \Vert + \Vert
  \mathbf{Q} \Vert$
\end{enumerate}
\end{mag_vectors}

\newpage
\subsection{Produs scalar}
\label{ch1:sec_vectors:dot_product}
\indent

\textit{Produsul scalar} a doi vectori, numit si \textit{produs interior}, este
una din cele mai folosite operații într-o aplicație 3D deoarere furnizează o
măsură a diferenței dintre direcțiile a doi vectori.
\newtheorem{def_dotproduct}{Definiția}
\begin{def_dotproduct}
Fiind dați doi vectori \textit{n}-dimensionali \textbf{P} și \textbf{Q},
produsul scalar al acestora, notat cu $\mathbf{P} \cdot \mathbf{Q}$, este
numărul real dat de formula
\begin{equation}
\label{eq:vec:8}
\mathbf{P} \cdot \mathbf {Q} = \sum_{i = 1}^{n} P_iQ_i.
\end{equation}
\end{def_dotproduct}
Această definiție spune că produsul scalar a doi vectori este dat de suma
produselor componentelor vectorilor. In $\mathbf{R^3}$ avem
\begin{equation}
\label{eq:vec:9}
\mathbf{P \cdot Q} = P_xQ_x + P_yQ_y + P_zQ_z.
\end{equation}
Produsul scalar mai poate fi exprimat și ca un produs de două matrici
\begin{equation}
\label{eq:vec:10}
\mathbf{P^TQ} = [P_1 \quad P_2 \quad \dots \quad P_n]
\begin{bmatrix}
Q_1 \\
Q_2 \\
\vdots \\
Q_n
\end{bmatrix},
\end{equation}

\newtheorem{dotproduct_cosine}[vector_arith]{Teorema}
\begin{dotproduct_cosine}
Fiind dați doi vectori \textit{n}-dimensionali \textbf{P} și \textbf{Q},
produsul lor scalar $\mathbf{P \cdot Q}$ satisface equația
\begin{equation}
\label{eq:vec:11}
\mathbf{P \cdot Q} = \Vert P \Vert \Vert Q \Vert \cos \alpha,
\end{equation}
unde $\alpha$ este unghiul dintre cei doi vectori.
\end{dotproduct_cosine}
Din această teoremă rezultă că doi vectori sunt perpendiculari dacă și numai
dacă produsul lor scalar este zero (funcția cosinus este zero la un unghi de 90
de grade). Pe baza semnului produsului scalar se poate face o observație legată
de cît de mult cei doi vectori indică direcții similare. Dacă considerăm un plan
care trece prin origine și care este perpendicular pe vectorul respectiv, atunci
orice vector care se situează pe aceeași parte a planului cu vectorul inițial va
avea un produs scalar pozitiv cu el și viceversa.

\newtheorem{dotproduct_properties}[vector_arith]{Teorema}
\begin{dotproduct_properties}
Fie un scalar \textbf{a} și trei vectori \textbf{P, Q, R}. Atunci, următoarele
proprietăți sunt adevărate :
\begin{enumerate}
  \item $\mathbf{P \cdot Q} = \mathbf{Q \cdot P}$ (comutativitatea produsului
  scalar)
  \item $(\mathit{a}\mathbf{P}) \cdot \mathbf{Q} = \mathit{a}(\mathbf{P \cdot
  Q})$ (asociativitatea înmulțirii cu scalari)
  \item $\mathbf{P} \cdot (\mathbf{Q + R}) = \mathbf{P \cdot Q + P \cdot R} $
  (distributivitatea produsului scalar față de adunarea vectorilor)
  \item $\mathbf{P \cdot P} = \Vert \mathbf{P} \Vert ^ 2$
  \item $\mathbf{\vert P \cdot Q \vert} \leq \Vert P \Vert \Vert Q \Vert$
\end{enumerate}
\end{dotproduct_properties}

Deseori apare situația în care trebuie să descompunem un vector \textbf{P} în
componente care sunt paralele și perpendiculare cu un alt vector \textbf{Q}
\todo{inserat figura + restul de formule care vin aici}

\subsection{Produs vectorial}
\label{ch1:sec_vectors:cross_product}
\indent

\textit{Produsul vectorial} a doi vectori tridimensionali întoarce un vector
care este perpendicular pe cei voi vectori multiplicați. Această proprietate are
multe utilizări într-o aplicație 3D, una din ele fiind o metodă de a calcula
normalul la suprafață intr-un anumit punct al ei, dîndu-se doi vectori distincți
și tangenți.

\newtheorem{crossproduct_def}[def_dotproduct]{Definiția}
\begin{crossproduct_def}
Se numește produs vectorial a doi vectori tridimensionali \textbf{P, Q} și se
notează cu \textbf{P $\times$ Q}, vectorul dat de formula
\begin{equation}
\label{eq:vec:12}
\mathbf{P \times Q} = \langle P_yQ_z - P_zQ_y, P_zQ_x - P_xQ_z, P_xQ_y - P_yQ_x
\rangle.
\end{equation}
\end{crossproduct_def}
O metodă utilă pentru reținerea acestei formule este de a calcula produsul
vectorial prin evaluarea pseudo-determinantului
\begin{equation}
\begin{vmatrix}
i & j & k \\
P_x & P_y & P_z \\
Q_x & Q_y & Q_z
\end{vmatrix},
\end{equation} unde \textit{i}, \textit{j}, \textit{k} sunt vectorii unitate
paraleli cu axele \textit{Ox, Oy, Oz}.

\indent
Ca și produsul scalar, produsul vectorial are semnificație trigonometrică.

\newtheorem{crossproduct_trig}[vector_arith]{Teorema}
\begin{crossproduct_trig}
Oricare ar fi doi vectori $\mathbf{P, Q \in R^3}$, produsul vectorial
$\mathbf{P \times Q}$ verifică ecuația
\begin{equation}
\label{eq:vec:13}
\Vert P \times Q \Vert = \Vert P \Vert \Vert Q \Vert \sin \alpha,
\end{equation} unde $\alpha$ reprezintă unghiul dintre vectorii \textbf{P, Q}.
\end{crossproduct_trig}
\todo {figura de inserat aici}

\newtheorem{crossproduct_properties}[vector_arith]{Teorema}
\begin{crossproduct_properties}
\label{cross_properties}
Oricare ar fi doi scalari $\mathit{a, b} \in \mathbf{R}$ si oricare trei vectori
$\mathbf{P, Q, R} \in \mathbf{R^3}$, următoarele proprietăți sunt verificate.
\begin{enumerate}
  \item $Q \times P = -(P \times Q)$ (anticomutativitate)
  \item $(\mathit{a}P) \times Q = \mathit{a}(P \times Q)$
  \item $P \times (Q + R) = P \times Q + P \times R$
  \item $P \times P = \langle 0, \quad 0, \quad 0 \rangle$
  \item $\Vert P \times Q \Vert ^ 2 = \Vert P \Vert ^ 2 \Vert Q \Vert ^ 2 - (P
  \cdot Q) ^ 2$ (identitatea Lagrange)
  \item produsul vectorial a doi vectori este nul dacă și numai dacă cei doi
  vectori sunt coliniari; dacă vectorii nu sunt coliniari, atunci norma
  produsului vectorial este egală cu aria paralelogramului construit pe
  reprezentanții cu origine comună ai celor doi vectori.
\end{enumerate}
\end{crossproduct_properties}

Ca și o consecință a teoremei \ref{cross_properties}, punctul 6, 
aria unui triunghi arbitrar ale cărui vertice sunt date de punctele
$\mathbf{V_0, V_1, V_2}$ poate fi calculată după formula
\begin{equation}
\label{eq:vec:cross_tri_area}
A = \frac{1}{2} \Vert (V_1 - V_0) \times (V_2 - V_0) \Vert
\end{equation}

\subsection{Dublu produs vectorial}
\label{ch1:sec_vectors:sub_double_cross_product}
\indent

Fiind dați vectorii $\mathbf{P, Q, R}$, vectorul \textbf{W} = $\mathbf{P
\times (Q \times R)}$ este dublul produs vectorial al acestor vectori.
Expresia dublului produs vectorial se reține mai ușor scrisă sub forma
pseudo-determinantului
\begin{equation}
\label{eq:vec:double_vector_product}
\mathbf{P \times (Q \times R)} = 
\begin{vmatrix}
Q & R \\
(P, Q) & (P, R)
\end{vmatrix}.
\end{equation}

\subsection{Produs mixt}
\label{ch1:sec_vectors:sub_mixed_product}
\indent

Fiind dați vectorii $\mathbf{P, Q, R}$, scalarul \textbf{W} = $\mathbf{P
\cdot (Q \times R)}$ se numește produsul mixt al acestor vectori. Dacă vectori
sunt necoplanari, atunci modulul produsului mixt este egal cu volumul
paralelipipedului ce se poate construi pe reprezentanții cu origine comună ai
celor trei vectori.
\todo{Figura de introdus aici}

\newtheorem{mixed_product}[vector_arith]{Teorema}
\begin{mixed_product}
Următoarele proprietăți sunt adevărate pentru produsul mixt a trei vectori
\textbf{P, Q, R}.
\begin{enumerate}
  \item $(P, Q \times R) = (R, Q \times P) = (Q, R \times P)$
  \item $(P, Q \times R) = -(P, R \times Q)$
  \item $(tP, Q \times R) = (P, tQ \times R) = (P, Q \times tR), \forall t \in
  \mathbf{R}$
  \item $(P_1 + P_2, Q \times R) = (P_1, Q \times R) + (P_2, Q \times R)$
  \item $(P \times Q, R \times S) =
  \begin{vmatrix}
  (P, R) & (P, S) \\
  (Q, R) & (Q, S)
  \end{vmatrix} 
  $ (identitatea Lagrange)
  \item $(P, Q \times R) = 0$ dacă și numai dacă
  \begin{itemize}
    \item[(i)] cel puțin unul dintre vectorii \textbf{P, Q, R} este nul;
    \item[(ii)] doi dintre vectori sunt coliniari;
    \item[(iii)] vectorii \textbf{P, Q, R} sunt coplanari. 
  \end{itemize}
\end{enumerate}
\end{mixed_product}

Produsul mixt a trei vectori \textbf{P, Q, R} poate fi calculat mai ușor prin
evaluarea determinantului 
$
\begin{vmatrix}
P_x & P_y & P_z \\
Q_x & Q_y & Q_z \\
R_x & R_y & R_z
\end{vmatrix}
$.

\subsection{Reprezentarea conceptului de vector in codul aplicației}
\label{ch1:sec_vectors:sub_vector_cpp_impl}
\indent

Odată introdusă reprezentarea matematică a conceptului de vector, este momentul
sa vedem cum este transpus acest concept în cod. Librăria \textbf{gfx\_lib}
conține clase pentru vectori cu 2, 3 și 4 componente, corespunzînd vectorilor
din $\mathbf{R^2, R^3, R^4}$. Clasele sunt clase template, fiind parametrizate
în funcție de tipul componentelor (numere intregi/reale).
Urmează cîteva fragmente de cod din implementarea clasei \textbf{vector3}, care
abstractizează un vector cu 3 componente din $\mathbf{R^3}$ (clasele
\textbf{vector2} si \textbf{vector4} sunt similare).

\todo{Codul sursa pt listing mai trebuie rearanjat / scurtat}
\lstinputlisting[
    caption = Reprezentarea in cod a conceptului de vector,
    label=lst:vec:vec4_class,
    linerange={1-141}]{./capitolul1/source_code/vector3.h.src}

Variabilele membre care reprezintă componentele vectorului sunt declarate în
partea publică a clasei. Am ales această opțiune deoarece a implementa
getteri/setteri ar fi complicat liziblitatea codului și ar fi insemnat 
și duplicare de cod. Clasa are un constructor default, care lasă componentele
neinițializate. Constructorul ar putea ințializa toate elementele la zero, dar
asta ar însemna timp suplimentar, care poate deveni semnificativ atunci cînd 
avem de-a face cu array-uri de dimensiuni mai mari. În plus, în majoritatea
cazurilor, imediat după inițializare, obiectele vor primi alte valori. Pentru
cazul în care se dorește inițializarea cu anumite valori specifice există mai mulți
constructori. Clasa oferă operatori și funcții non-membre pentru toate
operațiile cu vectori introduse în capitolele anterioare.

\todo{Mutat înainte de vectori ??}
\subsection{Spații vectoriale}
\label{ch1:sec_vectors:sub_vector_spaces}
Tra la la
\subsubsection{Grupuri}
\label{ch1:sec_vectors:sub_vector_spaces:sub_sub_groups}
\indent

Prin \textit{structură algebrică} înțelegem o mulțime pe care s-au definit un
număr finit de legi de compoziție și de relații, împreună cu proprietățile lor.
O lege de compiziție poate fi o operație unară, binară, \textit{n}-ară.
Înainte de a defini spațiile vectoriale este util să prezentăm noțiunile de grup
și câmp.

\indent
Fie \textit{G} o mulțime nevidă. O funcție definită pe $\mathit{G \times G}$ și
cu valori în \textit{G} se numește \textit{operație binară pe G}. O operație
binară pe \textit{G} se notează cu *, valoarea ei se notează cu $\mathit{g_1 *
g_2}$ și se citește „$\mathit{g_1}$ compus cu $\mathit{g_2}$”.

\newtheorem{def_group}[def_dotproduct]{Definiția}
\begin{def_group}
O mulțime G împreună cu o operație binară * pe G și care satisface condițiile :
\begin{enumerate}[(a)]
  \item $\forall g_1, g_2, g_3 \in G, \quad g_1 * (g_2 * g_3) = (g_1 * g_2) *
  g_3$ (asociativitate),
  \item $\exists e \in G, \quad \forall g \in G, \quad e * g = g * e = g$
  (element neutru),
  \item $\forall g \in G, \quad \exists g' \in G, \quad g * g' = g' * g = e$
  (element simetric) \end{enumerate} se numește grup.
\end{def_group}
Un grup se notează prin (G, *) soar doar prin G, caz în care operația se deduce
din context.
Dacă $\forall g_1, g_2 \in G, \quad g_1 * g_2 = g_2 * g_1$ (comutativitate),
atunci grupul G se numește \textit{grup comutativ sau abelian}.

\newtheorem{def_isomorph}[def_dotproduct]{Definiția}
\begin{def_isomorph}
Fie $(G_1, *), \quad (G_2, \circ)$ două grupuri. O funcție $\phi : G_1
\rightarrow G_2$ care verifică relația $\phi(g_1 * g_2) = \phi(g_1) \quad \circ \quad
\phi(g_2), \quad \forall g_1, g_2 \in G_1$ se numește homomorfism. Un
homomorfism bijectiv se numește izomorfism.
\end{def_isomorph}

\newtheorem{def_field}[def_dotproduct]{Definiția}
\begin{def_field}
O mulțime K, împreună cu două aplicații ale lui $K \times K \rightarrow K$,
numite adunare, respectiv înmulțire și care verifică cerințele :
\begin{enumerate}
  \item adunarea determină pe K o structura de grup abelian
  \item înmulțirea determină pe K - $\lbrace 0 \rbrace$ o structură de grup
  \item înmulțirea este distributivă față de adunare
\end{enumerate} se numește corp. Un corp pentru care operația de înmulțire
verifică și condiția de comutativitate se numește câmp.
\end{def_field}

\subsubsection{Spații vectoriale}
\label{ch1:sec_vectors:sub_vector_spaces:sub_sub_vectorspaces}
Structura de spațiu vectorial constă dintr-un grup aditiv comutativ \textbf{V},
un câmp \textbf{K} și o înmulțire definită pe $\mathbf{K \times V} \rightarrow
\mathbf{V}$, care verifică patru axiome.
\newtheorem{def_vectorspaces}[def_dotproduct]{Definiția}
\begin{def_vectorspaces}
Mulțimea \textbf{V} se numește spațiu vectorial peste câmpul \textbf{K} daca
admite
\begin{enumerate}
  \item o structură de grup comutativ, notată aditiv, $(v, w) \rightarrow v
  + w$
  \item o funție $f : \mathbf{K \times V} \rightarrow V, \quad f(k, v) = kv$,
  astfel încât $\forall k, l \in \mathbf{K}, \quad \forall v, w \in \mathbf{V}$
  să avem : \begin{center}
  $1v = v$ \\ $k(lv) = (kl)v$ \\ $(k + l)v = kv + lv$ \\ $k(v + w) = kv + kw$
  \end{center}  
\end{enumerate}
\end{def_vectorspaces}
Elementele lui \textbf{V} se numesc vectori, elementele lui \textbf{K}
se numesc scalari, iar aplicația \textit{f} înmulțirea vectorilor cu scalari.

Fie \textbf{V} un spațiu vectorial pe câmpul \textbf{K} și \textbf{S} o
submulțime a sa.
\newtheorem{th_liniar_independent}[def_dotproduct]{Definiția}
\begin{th_liniar_independent}
Mulțimea \textbf{S} se numește liniar independentă dacă pentru orice alegere a n
elemente din \textbf{S}, respectiv \textbf{K}, relația $\sum_{i = 1}^{n}s_ik_i$
implică faptul că scalarii $k_1 = k_2 = \dots = k_n = 0$. În caz contrar se
spune că multimea \textbf{S} este liniar dependentă.
\end{th_liniar_independent}

\newtheorem{th_liniar_comb}[def_dotproduct]{Definiția}
\begin{th_liniar_comb}
Spunem că un vector $x \in \mathbf{V}$ este o combinație liniară de vectorii
$v_1, v_2, \dots v_n$ dacă există scalarii $k_1, k_2, \dots, k_n \in \mathbf{K}$
astfel încât $x = \sum_{i = 1}^{n} k_iv_i$.
\end{th_liniar_comb}

\newtheorem{th_generator}[def_dotproduct]{Definiția}
\begin{th_generator}
Spunem că vectorii $v_1, v_2, \dots, v_n$ formează un sistem de generatori
pentru spațiul vectorial \textbf{V} dacă $\forall x \in \mathbf{V}, \quad
\exists k_1, k_2, \dots, k_n \in \mathbf{K}$ astfel încât $x = \sum_{i
= 1}^{n}v_ik_i$.
\end{th_generator}

\newtheorem{th_vector_basis}[def_dotproduct]{Definiția}
\begin{th_vector_basis}
Fie \textbf{V} un spațiu vectorial peste un câmp \textbf{K}. Un sistem $B = (
e_1, e_2, \dots, e_n )$ de vectori din \textbf{V}, $1 \leq i \leq n$, se 
numește bază a lui \textbf{V}, dacă :
\begin{enumerate}
  \item orice vector din \textbf{V} se poate scrie ca o combinație liniară de
  vectorii bazei \textbf{B} (\textbf{B} este un sistem de generatori pentru
  spațiul vectorial \textbf{V}).
  \item \textbf{B} este un sistem de vectori liniar independent.
\end{enumerate}
\end{th_vector_basis}

Dacă pentru orice pereche de elemente ale unei baze, de forma $(e_i, e_j), i
\neq j$, se verifică relația $e_i \cdot e_j = 0$, atunci baza se numește
ortogonală.

Fie simbolul $\delta$ al lui Kronecker, definit ca 
\begin{equation}
\delta_{ij} =
\begin{cases}
1, & \text{dacă } i = j, \\
0, & \text{dacă } i \neq j 
\end{cases}
\end{equation}
Atunci, daca pentru orice pereche de elemente ale unei baze, de forma $(e_i,
e_j)$, este adevarată relația $e_i \cdot e_j = \delta_{ij}$, baza se numește
ortonormală.

\subsubsection{Ortogonalizare Gram-Schmidt}
\label{ch1:sec_vectors:sub_vector_spaces:sub_sub_gram_schmidt}
Procedeul de ortogonalizare Gram-Schmidt este un algoritm care primește ca și
input un set de vectori liniar independenți și produce ca și rezultat un set de
vectori ortogonali.

\begin{algorithm}
\caption{Ortogonalizare Gram-Schmidth}
\label{alg:gram_schmidt}
\begin{algorithmic}
\STATE Input : un set de vectori liniar independenți $v_1, v_2, \dots v_n$.
\STATE Output : un set de vectori ortonogonali $v'_1, v'_2, \dots v'_n$.
\FOR{$i=1$ \TO $n$}
\STATE $v'_i \leftarrow v_i$
\FOR{$j=1$ \TO $i$}
\STATE $v'_i \leftarrow v'_i - \frac{(v'_j, v_i)}{\Vert v'_j \Vert ^ 2}v'_j$
\ENDFOR 
\ENDFOR
\end{algorithmic}
\end{algorithm}
Algoritmul \ref{alg:gram_schmidt} este instabil numeric. Este posibil ca
vectorii rezultați să nu mai fie ortogonali, din cauza erorilor de rotunjire. 
Procesul Gram-Schmidt poate fi stabilizat printr-o mică modificare, această
versiune fiind cunoscută sub denumirea de \textbf{ortogonalizare Gram-Schmidt
modificată}. Timpul de execuție al algoritmului este \textbf{O(2mn)}, unde n
reprezintă numărul de vectori din setul de input, iar m dimensionalitatea vectorilor
\cite{GolubVanLoan96}.

\begin{algorithm}
\caption{Ortogonalizare Gram-Schmidt modificată}
\label{alg:modified_gram_schmidt}
\begin{algorithmic}
\STATE Input : un set de vectori liniar independenți $v_1, v_2, \dots v_n$.
\STATE Output : un set de vectori ortonormali $v'_1, v'_2, \dots v'_n$.
\FOR{$i=1$ \TO $n$}
\STATE $v'_i \leftarrow v_i$
\FOR{$j=1$ \TO $i$}
\STATE $v'_i \leftarrow v'_i - \frac{(v'_j, v_i)}{\Vert v'_j \Vert ^ 2}v'_j$
\ENDFOR 
\STATE $v'_i = v'_i / \Vert v'_i \Vert$ \COMMENT{normalizează vectorul rezultat}
\ENDFOR
\end{algorithmic}
\end{algorithm}
