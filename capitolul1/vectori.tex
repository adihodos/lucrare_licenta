\section{Vectori}
\label{ch1:sec_vectors}
\indent

Vectorii au o importanță fundamentală pentru orice aplicație 3D. Ei sunt
utilizați pentru a reprezenta puncte in spațiu, precum locațiile obiectelor sau
verticele unei meșe de triunghiuri. De asemenea sunt utilizați pentru a
reprezenta direcții în spațiu, precum orientarea camerei sau normalii la
suprafață într-o meșă de triunghiuri.

\subsection{Implementarea conceptului de vector în codul aplicației}
\label{ch1:sec_vectors:sub_properties}
\indent

Deoarece vectorii sunt un concept studiat în liceu/facultate, lucrarea nu mai 
face o prezentare teoretică a conceptului. În schimb, acest capitol detaliază 
modul în care acest concept este transpus în aplicație.

Librăria \textbf{gfx\_lib} conține clase pentru vectori cu 2, 3 și 4 componente, 
corespunzînd vectorilor din $\mathbf{R^2, R^3, R^4}$. Clasele sunt clase template, 
fiind parametrizate în funcție de tipul componentelor (numere intregi/numere 
reale cu simpla/dublă precizie).
Acest lucru permite folosirea de vectori ale căror elemente sunt numere reale, 
stocate folosind reprezentare în virgulă mobila cu simplă/dublă precizie, fără a
fi nevoie de duplicarea codului existent. În marea majoritate a cazurilor, 
vectorii cu elemente reprezentate în virgulă mobilă cu simplă precizie sunt 
alegerea cea mai buna. Sunt însă situații în care utilizarea reprezentării în 
virgulă mobilă cu dublă precizie este de preferat.
Fragmentele de cod prezintă clasa \textbf{vector3}, care
abstractizează un vector cu 3 componente din $\mathbf{R^3}$. Clasele
\textbf{vector2} si \textbf{vector4}, corespunzând vectorilor din 
$\mathbf{R^2} \text{ și } \mathbf{R^4}$, sunt similare ca și design și 
funcționalitate.

\begin{lstlisting}[
    title={Cod C++}, 
    xleftmargin=2pt,
    abovecaptionskip=2pt,
    belowcaptionskip=2pt,
    frame=single]
template<typename real_t>
class vector3 {
public:
enum {
    is_floating_point = math::fp_test<real_t>::result
};

union { struct {
        real_t x_;
        real_t y_;
        real_t z_;
    };
    real_t elements_[3];
};
\end{lstlisting}

Membrul enumerator \textit{is\_floating\_point} este o constantă determinată la
compilare, al carei rol este de a permite optimizarea unor operații în funcție
de tipul componentelor (integer, float sau double). De exemplu, pentru valori
floating point împărțirea nu se face în mod direct, ci se calculează inversul
împărțitorului și se inmulțește deîmpărțitul cu această valoare.
Motivul este eficiența : înmulțirea a două elemente floating point consumă mai puține
cicluri de procesor față de o împărțire. Înainte de efectuarea operației de
împărțire se aplică o transformare asupra împărțitorului. Pentru valori numere
întregi, transformarea este funcția identică ($f(x) = x$) :
\begin{lstlisting}[
    title={Cod C++}, 
    xleftmargin=2pt,
    abovecaptionskip=2pt,
    belowcaptionskip=2pt,
    frame=single]
template<typename real_t, bool is_floating_point = false>
struct transform_dividend_for_division {
    static real_t transform(real_t dividend) {
        return dividend;
    }
};
\end{lstlisting} iar împărțirea se face în mod direct :
\begin{lstlisting}[
    title={Cod C++}, 
    xleftmargin=2pt,
    abovecaptionskip=2pt,
    belowcaptionskip=2pt,
    frame=single]
template<typename real_t, bool is_floating_point = false>
struct divide_helper {
    static real_t divide(real_t divided, real_t dividend) {
        return divided / dividend;
    }
};
\end{lstlisting}

Pentru valori floating point, transformarea este funcția inversă
($f(x) = \frac{1}{x}$) :
\begin{lstlisting}[
    title={Cod C++}, 
    xleftmargin=2pt,
    abovecaptionskip=2pt,
    belowcaptionskip=2pt,
    frame=single]
template<typename real_t>
struct transform_dividend_for_division<real_t, true> {
    static real_t transform(real_t dividend) {
        return real_t(1) / dividend;
    }
};
\end{lstlisting} iar rezultatul operației este calculat prin înmulțire cu inversa :
\begin{lstlisting}[
    title={Cod C++}, 
    xleftmargin=2pt,
    abovecaptionskip=2pt,
    belowcaptionskip=2pt,
    frame=single
    ]
template<typename real_t>
struct divide_helper<real_t, true> {
    static real_t divide(real_t divided, real_t inv) {
        return divided * inv;
    }
};
\end{lstlisting}

Variabilele membre care reprezintă componentele vectorului sunt declarate în
partea publică a clasei. Am ales această opțiune deoarece a implementa
getteri/setteri ar fi complicat liziblitatea codului și ar fi insemnat 
și duplicare de cod. Clasa are un constructor default, care lasă componentele
neinițializate. Constructorul ar putea ințializa toate elementele la zero, dar
asta ar însemna timp suplimentar, care poate deveni semnificativ atunci cînd 
avem de-a face cu array-uri de dimensiuni mai mari. În plus, în majoritatea
cazurilor, imediat după inițializare, obiectele vor primi alte valori. Pentru
cazul în care se dorește inițializarea cu anumite valori specifice există mai 
mulți constructori supraîncărcați.
\begin{lstlisting}[
    title={Cod C++}, 
    xleftmargin=2pt,
    abovecaptionskip=2pt,
    belowcaptionskip=2pt,
    frame=single]
template<typename real_t>
class vector {
    vector3() {}
    vector3(real_t x, real_t y, real_t z) 
        : x_(x), y_(y), z_(z) {}
    vector3(const real_t* input, size_t count);
};
\end{lstlisting}
Clasa implementează toate operațiile aritmetice ce se pot aplica vectorilor :
\begin{easylist}[itemize]
\ListProperties(Numbers1=a)
@ înmulțirea/împărțirea cu scalari 
\begin{lstlisting}
template<typename real_t>
inline
gfx::vector3<real_t>&
gfx::vector3<real_t>::operator *=(real_t k) {
    x_ *= k; y_ *= k; z_ *= k;
    return *this;
}

template<typename real_t>
inline
gfx::vector3<real_t>&
gfx::vector3<real_t>::operator /=(real_t k) {
    using namespace implementation_details;
    const real_t dividend = transform_dividend_for_division<
        real_t, 
        is_floating_point
    >::transform(k);

    divide_helper<
        real_t,
        is_floating_point
    > div_helper;

    x_ = div_helper(x_, dividend);
    y_ = div_helper(y_, dividend);
    z_ = div_helper(z_, dividend);
    return *this;
}
\end{lstlisting}
@ adunarea și scăderea vectorilor
\begin{lstlisting}
template<typename real_t>
inline gfx::vector3<real_t>&
gfx::vector3<real_t>::operator+=(
    const vector3<real_t>& rhs
    ) 
{
    x_ += rhs.x_; y_ += rhs.y_; z_ += rhs.z_;
    return *this;
}
\end{lstlisting}

@ determinarea magnitudinii și normalizarea unui vector
\begin{lstlisting}
template<typename real_t>
inline real_t
gfx::vector3<real_t>::sum_components_squared() const {
    return x_ * x_ + y_ * y_ + z_ * z_;
}

template<typename real_t>
inline real_t
gfx::vector3<real_t>::magnitude() const {
    return std::sqrt(sum_components_squared());
}

template<typename real_t>
inline gfx::vector3<real_t>& gfx::vector3<real_t>::normalize() {
    float magn(magnitude());
    if (math::is_zero(magn)) {
        x_ = y_ = z_ = real_t(0);
    } else {
        *this /= magn;
    }
    return *this;
}
\end{lstlisting}

@ produs scalar și produs vectorial

\begin{lstlisting}
template<typename real_t>
inline real_t gfx::dot_product(
    const gfx::vector3<real_t>& lhs, 
    const gfx::vector3<real_t>& rhs
    ) {
    return lhs.x_ * rhs.x_ + lhs.y_ * rhs.y_ + lhs.z_ * rhs.z_;
}

template<typename real_t>
inline gfx::vector3<real_t> gfx::cross_product(
    const gfx::vector3<real_t>& lhs, 
    const gfx::vector3<real_t>& rhs
    ) {
    return vector3<real_t>(
        lhs.y_ * rhs.z_ - lhs.z_ * rhs.y_,
        lhs.z_ * rhs.x_ - lhs.x_ * rhs.z_,
        lhs.x_ * rhs.y_ - lhs.y_ * rhs.x_
        );
}
\end{lstlisting}

@ proiecția unui vector pe un alt vector

\begin{lstlisting}
template<typename real_t>
inline
gfx::vector3<real_t>
gfx::project_vector_on_vector(
    const gfx::vector3<real_t>& lhs, 
    const gfx::vector3<real_t>& rhs
    )
{
    return (dot_product(lhs, rhs) 
            / rhs.sum_components_squared()) * rhs;
}

template<typename real_t>
inline
gfx::vector3<real_t>
gfx::project_vector_on_unit_vector(
    const gfx::vector3<real_t>& lhs, 
    const gfx::vector3<real_t>& rhs
    )
{
    dot_product(lhs, rhs) * rhs;
}
\end{lstlisting}

\end{easylist}

\subsubsection{Ortogonalizare Gram-Schmidt}
\label{ch1:sec_vectors:sub_gram_schmidt}
Procedeul de ortogonalizare Gram-Schmidt este un algoritm care primește ca și
input un set de vectori liniar independenți și produce ca și rezultat un set de
vectori ortogonali.

\begin{algorithm}
\caption{Ortogonalizare Gram-Schmidth}
\label{alg:gram_schmidt}
\begin{algorithmic}
\STATE Input : un set de vectori liniar independenți $v_1, v_2, \dots v_n$.
\STATE Output : un set de vectori ortonogonali $v'_1, v'_2, \dots v'_n$.
\FOR{$i=1$ \TO $n$}
\STATE $v'_i \leftarrow v_i$
\FOR{$j=1$ \TO $i$}
\STATE $v'_i \leftarrow v'_i - \frac{(v'_j, v_i)}{\Vert v'_j \Vert ^ 2}v'_j$
\ENDFOR 
\ENDFOR
\end{algorithmic}
\end{algorithm}
Algoritmul \ref{alg:gram_schmidt} este instabil numeric. Este posibil ca
vectorii rezultați să nu mai fie ortogonali, din cauza erorilor de rotunjire. 
Procesul Gram-Schmidt poate fi stabilizat printr-o mică modificare, această
versiune fiind cunoscută sub denumirea de \textbf{ortogonalizare Gram-Schmidt
modificată}. Timpul de execuție al algoritmului este \textbf{O(2mn)}, unde n
reprezintă numărul de vectori din setul de input, iar m dimensionalitatea vectorilor
\cite{GolubVanLoan96}.

\begin{algorithm}
\caption{Ortogonalizare Gram-Schmidt modificată}
\label{alg:modified_gram_schmidt}
\begin{algorithmic}
\STATE Input : un set de vectori liniar independenți $v_1, v_2, \dots v_n$.
\STATE Output : un set de vectori ortonormali $v'_1, v'_2, \dots v'_n$.
\FOR{$i=1$ \TO $n$}
\STATE $v'_i \leftarrow v_i$
\FOR{$j=1$ \TO $i$}
\STATE $v'_i \leftarrow v'_i - \frac{(v'_j, v_i)}{\Vert v'_j \Vert ^ 2}v'_j$
\ENDFOR 
\STATE $v'_i = v'_i / \Vert v'_i \Vert$ \COMMENT{normalizează vectorul rezultat}
\ENDFOR
\end{algorithmic}
\end{algorithm}
