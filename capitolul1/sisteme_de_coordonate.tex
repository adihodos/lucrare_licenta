\section{Sisteme de coordonate in $\mathbf{R}$, $\mathbf{R^2}$ si
$\mathbf{R^3}$}
\label{ch1:sec_coord_sys}

\subsection{Geometria analitica pe o dreapta}
\label{ch1:sec_coord_sys:geometrie_dreapta}
\indent

O dreaptă D pe care s-a luat un punct \textbf{O}, o unitate de măsură \textit{u}
și un sens de parcurs se numește \emph{axă}. Un număr real x asociat unui punct 
M de pe o axă se numește coordonata punctului M față de axa considerată. Dacă
alegem originea \textbf{O} și un sens de parcurs pe axă, se numește \textit{sens
pozitiv}, dacă parcurgând segmentul de la punctul care marcheaza originea
(\textbf{O}) la un alt punct de pe axă (\textbf{P}), \textbf{P} are coordonată
pozitivă. Sensul negatic este cel opus sensului pozitiv. Rezultă deci, ca
coordonata unui punct depinde de origine, sens și de unitatea de măsură
considerată.
Astfel, dacă un punct \textbf{M} are coordonata \textit{x} pe o axă
\textit{D} și îi schimbăm sensul, față de noua axă $\mathit{D'}$ coordonata sa
este $\mathit{X' = -x}$. Dacă schimbăm pe \textit{D} unitatea de măsură, anume o
luăm de $\mathit{|h|}$ ori mai mică, $\mathit{h \neq 0}$, obținem o nouă axă
$\mathit{D''}$, față de care coordonata $\mathit{X''}$ a lui $\mathit{M}$
se scrie $\mathit{X'' = hx}$. În fine, dacă la axa \textit{D} îi schimbăm
originea \textit{O} în punctul $\mathit{O'}$ de coordonată \textit{a}, obținem o
nouă axă $\mathit{D'''}$, față de care coordonata $\mathit{X'''}$ a punctului
\textit{M} se scrie $\mathit{X''' = x - a}$. Considerând acum toate cele trei
operații făcute asupra axei \textit{D}, obținem o nouă axă $\mathit{D^*}$, iar
coordonata $\mathit{x^*}$ a punctului \textit{M} față de noua axă are expresia
$\mathit{x^* = hx - a}$.

\subsection{Sisteme de coordonate în plan (repere plane)}
\label{ch1:sec_coord_sys:cartesian_plane}
\indent

Planul reprezintă mulțimea tuturor tuplelor ordonate de două numere reale, adică
$P = \lbrace (x, y) \quad \vert \quad x \in \mathbf{R}, \quad y \in \mathbf{R}
\rbrace $. Unei tuple de numere reale i se asociază un punct din plan și numai
unul. Considerăm două axe, \textit{D} și \textit{D'}, perpendiculare, cu aceeași
origine \textit{O} și aceeași unitate. Dreapta \textit{D} se numește
\textit{dreapta absciselor} iar coordonata x \textit{abscisă}. Dreapta
\textit{D'} se numește \textit{dreapta ordonatelor} iar coordonata y
\textit{ordonată}. Sistemul $(D, D')$ se numește \textit{sistem de coordonate
carteziene}. Deoarece cele două drepte sunt perpendiculare, sistemul
$\mathit{DOD'}$ se numește rectangular sau \textit{reper cartezian rectangular}.
Dacă cele două axe fac  între ele un unghi diferit de $\frac{\pi}{2}, \pi$,
reperul se numește \textit{oblic} sau general. 

\subsection{Sisteme de coordonate în spațiu (repere spațiale)}
\label{ch1:sec_coord_sys:space}
\indent

Spațiul cu trei dimensiuni $\mathbf{R^3}$ se definește ca fiind mulțimea
tuplelor ordonate de trei numere reale, adică $S = \lbrace (x, y, z) \quad
\vert \quad x \in \mathbf{R}, \quad y \in \mathbf{R}, \quad z \in \mathbf{R}
\rbrace$. Unui tuple de trei numere reale i se asociază un singur punct din
spațiul tridimensional și numai unul. Considerăm în spațiu trei axe \textit{Ox},
\textit{Oy} și \textit{Oz}, cu aceeași origine \textit{O}, deci concurente, cu
aceeași unitate de măsură, astfel încît axa \textit{Ox} este perpendiculară pe
planul format de \textit{Oz} și \textit{Oy} în punctul \textit{O}, axele
\textit{Oy} și \textit{Oz} fiind perpendiculare între ele. Cele trei axe
formează un \textit{sistem cartezian trirectangular}, sau \textit{reper
trirectangular Oxyz}. Axele \textit{Ox, Oy, Oz} se numesc \textit{axe de
coordonate}, iar planele \textit{Oxy, Oyz, Ozx} se numesc \textit{plane de
coordonate}.
