\section{Quaternioni}
\label{ch1:sec_quaternions}
\indent

Quaternionii au fost descoperiți de Sir William Rowan Hamilton in secolul al
XIX-lea. Acesta căuta un mod de a reprezenta numerele complexe in dimensiuni
superioare. De la descoperirea lor, matematicienii au arătat că quaternionii pot
fi utilizați pentru a roti puncte în jurul unor axe arbitrare.
Quaternionii au o serie de avantaje față de matrici. De exemplu, pentru
stocarea unui quaternion sunt nevoie doar de patru componente, concatenarea lor
implică mai puține operații aritmetice și sunt mai ușor de interpolat pentru a
produce animații. Utilizarea lor în grafica 3D a fost pionierată de Shoemake.

\todo{Schimbat titlul}
\subsection{Proprietățile quaternionilor}
\label{ch1:sec_quaternions:sub_properties}
\indent

Mulțimea quaternionilor, cunoscută în matematică sub denumirea de \textit{inelul
quaternionilor Hamiltonieni} și notată cu $\mathfrak{H}$ poate fi văzută ca
un spațiu vectorial patru dimensional, ale cărui elemente sunt tuple de 4
numere reale, de forma 
\begin{equation}
q = \langle w, x, y, z\rangle = w + x\mathit{i} + y\mathit{j} + z\mathit{k}.
\end{equation}
Un quaternion mai poate fi scris și ca 
\begin{equation}
q = \lbrack s, v \rbrack
\end{equation} unde s este partea scalară, iar v este partea vectorială.
Hamilton a numit quaternionii ai căror componentă scalară este egală cu zero
\textit{quaternioni puri}.
\newpage
Conceptual, implementarea quaternionului arată în felul următor :
\begin{lstlisting}[]
template<typename real_t>
class quaternion {
public :
    union {
        struct {
            real_t w_;
            real_t x_;
            real_t y_;
            real_t z_;
        };
        real_t elements_[4];
    };

    quaternion();
    quaternion(real_t w, real_t x, real_t y, real_t z);
    quaternion(const real_t* init_data);
    ...
};
\end{lstlisting}

\subsection{Adunarea quaternionilor}
\label{ch1:sec_quaternions:sub_add}
Quaternionii se adună și se scad pe componente, ca și vectorii.
Fiind dați doi quaternioni, $q_1 \text{ și } q_2$, suma lor este quaternionul
\begin{equation}
q = q_1 + q_2 = \lbrack s_1 + s_2, v_1 + v_2 \rbrack,
\end{equation} sau echivalent
\begin{equation}
q = q_1 + q_2 = \langle w_1 + w_2, \quad (x_1 + x_2)i, \quad (y_1 + y_2)j, 
\quad (z_1 + z_2)k \rangle.
\end{equation}

\noindent

Proprietățile adunării quaternionilor :
\begin{enumerate}
    \item Adunarea este comutativă : 
    $q_1 + q_2 = q_2 + q_1, \quad \forall q_1, q_2 \in \mathfrak{H}$
    \item Adunarea este asociativă :
    $(q_1 + q_2) + q_3 = q_1 + (q_2 + q_3), \quad \forall q_1, q_2, q_3
    \in \mathfrak{H}$
    \item Element neutru : $\exists q' = \langle 0, 0, 0, 0 \rangle \in 
    \mathfrak{H} \text{ a. î. } q + q' = q' + q = q, \quad \forall q \in
    \mathfrak{H}$ 
    \item Orice quaternion are un opus : $\forall q \in \mathfrak{H} \quad
    \exists q' = (-q) \in \mathfrak{H} \quad \text{ a.î. } q + q' = q' + q = 
    \langle 0, 0, 0, 0 \rangle$
\end{enumerate}

\subsection{Magnitudinea unui quaternion}
\label{ch1:sec_quaternions:sub_magnitude}
Similar cu vectorii, magnitudinea unui quaternion este dată de formula
\begin{equation}
\Vert q \Vert = \sqrt{w^2 + x^2 + y^2 + z^2}.
\end{equation}
Un quaternion care are o magnitudine de unu se spune că este un
\textit{quaternion unitate}. Componentele unui quaternion unitate verifică
relația
\begin{equation}
w^2 + v \cdot v = 1.
\end{equation}
Similar vectorilor, un quaternion cu o magnitudine
non-zero poate fi adus la lungime unitate prin operația de normalizare, adică
\begin{equation}
\hat{q} = \frac{q}{\Vert q \Vert}.
\end{equation}
Fie q și r quaternioni din $\mathfrak{H}$. Se verifică următoarele relații :
\begin{enumerate}
    \item $\Vert q \Vert ^ 2 = w^2 + v \cdot v$
    \item $\Vert q \Vert = 0 \iff q = \langle 0, 0, 0, 0 \rangle$
    \item $\Vert \mathit{a}q \Vert = \vert \mathit{a} \vert \Vert q \Vert$
    \item $\Vert q + r \Vert \leq \Vert q \Vert + \Vert r \Vert$
\end{enumerate}

\subsection{Produs scalar}
\label{ch1:sec_quaternions:sub_scalar_product}
Produsul scalar a doi quaternioni este similar cu produsul scalar a doi vectori,
fiind suma produselor componentelor celor doi operanzi. Fie $q_{1} \text{ și }
q_{2}$ doi quaternioni din $\mathfrak{H}$. Atunci produsul scalar al lor este dat de
formula
\begin{equation}
q_{1} \cdot q_{2} = \sum_{i = 1}^{4} q_{1i}q_{2i} 
\end{equation} sau echivalent
\begin{equation}
q_{1} \cdot q_{2} = w_{1}w_{2} + x_{1}x_{2} + y_{1}y_{2} + z_{1}z_{2}.
\end{equation}
Produsul scalar mai verifică și următoarele relații :
\begin{enumerate}
\item $q_1 \cdot q_2 = q_2 \cdot q_1$
\item $q \cdot q = \Vert q \Vert ^ 2$
\item $q_1 \cdot q_2 = 1 \Rightarrow q_1 \perp q_2$
\item $(\mathit{a}q_1) \cdot q_2 = \mathit{a}(q_1 \cdot q_2)$
\end{enumerate}
\todo{Verificat chestia asta din VanVerth}
Produsul scalar a doi quaternioni dă o masură a diferenței dintre rotațiile
aplicate de cei doi quaternioni. Cand acesta ia valori apropiate de 1, cei doi
quaternioni aplică rotații similare. Cand ia valori apropiate de -1, rotațiile
aplicate sunt foarte diferite.

\subsection{Înmulțirea quaternionilor}
Fiind dați doi quaternioni
\begin{IEEEeqnarray*}{rCl}
q_{1} = w_{1} + v_{1} = w_{1} + x_{1}\mathit{i} + y_{1}\mathit{j} 
                         + z_{1}\mathit{k},\\
q_{2} = w_{2} + v_{2} = w_{2} + x_{2}\mathit{i} + y_{2}\mathit{j}
                        + z_{2}\mathit{k},
\end{IEEEeqnarray*}
produsul lor este dat de
\begin{IEEEeqnarray*}{rCl}
q_{1}q_{2} = (w_{1}w_{2} - x_{1}x_{2} - y_{1}y_{2} - z_{1}z_{2})\\
+ (w_{1}x_{2} + x_{1}w_{2} + y_{2}z_{2} - z_{1}y_{2})\mathit{i}\\
+ (w_{1}y_{2} - x_{1}z_{2} + y_{1}w_{2} + z_{1}x_{2})\mathit{j}\\
+ (w_{1}z_{2} + x_{1}y_{2} - y_{1}x_{2} + z_{1}w_{2})\mathit{k}\IEEEyesnumber
\end{IEEEeqnarray*} sau echivalent
\begin{equation}
q_{1}q_{2} = w_{1}w_{2} - v_{1} \cdot v_{2} + w_{1}v_{2} + w_{2}v_{1} +
v_{1} \times v_{2}.
\end{equation}
Se poate observa că produsul a doi quaternioni este necomutativ, deoarece
conține un produs vectorial în componența sa.

Proprietățile înmulțirii quaternionilor.
\begin{enumerate}
    \item $q_{1}q_{2} \neq q_{2}q_{1}, \quad \forall q_{1}, q_{2} \in
    \mathfrak{H}$
    \item $q_{1}(q_{2}q_{3}) = (q_{1}q_{2})q_{3}, \quad \forall q_{1}, q_{2},
    q_{3} \in \mathfrak{H}$
    \item $\exists q_{I} \in \mathfrak{H} \text{ a.î. } qq_{I} = q_{I}q = q,
    \forall q \in \mathfrak{H}$
    \item $\exists q^{-1} \in \mathfrak{H} \text{ a.î. } qq^{-1} = q^{-1}q =
    q_{I}, \quad \forall q \in \mathfrak{H}$
    \item $q_{1}(q_{2} + q_{3}) = q_{1}q_{2} + q_{1}q_{3} \quad \forall q_{1},
    q_{2}, q_{3} \in \mathfrak{H}$
\end{enumerate}

\subsection{Conjugata unui quaternion}
Ca și numerele complexe, quaternionii au conjugate.
Dacă $q = s + v$ este un quaternion din $\mathfrak{H}$, atunci conjugatul său este
quaternionul
\begin{equation}
\bar{q} = s - v = s - (x\mathit{i} + y\mathit{j} + z\mathit{k}).
\end{equation}
Dacă efectuăm produsul între un quaternion și conjugatul său, obținem :
\[
q\bar{q} = w\bar{w} - v \cdot \bar{v} + w\bar{v} + \bar{w}v + v \times \bar{v}.
\]
Cum $v = -\bar{v} \text{ și } w = \bar{w}$ rezultă că $q\bar{q} = w^2 +
v\cdot v$, adică există identitatea : 
\begin{equation}
q\bar{q} = \bar{q}q = q \cdot q = \Vert q \Vert ^ 2.
\end{equation}

\subsection{Inversa unui quaternion}
Inversul unui quaternion non-zero $q \in \mathfrak{H}$ este quaternionul dat de 
formula
\begin{equation}
\label{eq:quat_inverse}
q^{-1} = \frac{\bar{q}}{\Vert q \Vert ^ 2}.
\end{equation}
Dacă q este un \textit{quaternion de magnitudine unitate}, formula
\eqref{eq:quat_inverse} poate fi redusă la
\begin{equation}
q^{-1} = \bar{q}.
\end{equation}

\subsection{Rotirea vectorilor}
