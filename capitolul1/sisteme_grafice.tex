\chapter{Fundamentele unui sistem grafic}
\label{ch1}

Sisteme de coordonate, vectori, matrici și transformări sunt concepte care apar
la fiecare pas într-o aplicație 3D. Ele pot fi întîlnite nu numai în motoarele
grafice dar și în cele de sunet sau fizică. Luarea unui model creat cu un anumit pachet
de modelare 3D și transpunerea lui în spațiul scenei, poziționarea camerei
pentru a observa scena, afișarea verticelor și triunghiurilor care compun un model,
toate acestea sunt procese care se folosesc de conceptele menționate anterior.

\todo{TODO - ??? - e ciudata formularea asta}
